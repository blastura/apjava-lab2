\documentclass[a4paper, 12pt]{article}
\usepackage[swedish]{babel}
\usepackage[utf8]{inputenc}
\usepackage{verbatim}
\usepackage{fancyhdr}
\usepackage{graphicx}
\usepackage{parskip}
% Include pdf with multiple pages ex \includepdf[pages=-, nup=2x2]{filename.pdf}
\usepackage[final]{pdfpages}
% Place figures where they should be
\usepackage{float}

% vars
\def\title{RSS}
\def\preTitle{Laboration 2}
\def\kurs{Applikationsprogrammering i Java, HT-08}

\def\namn{Anton Johansson}
\def\mail{dit06ajn@cs.umu.se}
 \def\pathtocode{$\sim$dit06ajn/edu/apjava/lab2}

\def\handledareEtt{Johan Eliasson, johane@cs.umu.se}
\def\handledareTva{Tor Sterner-Johansson, tors@cs.umu.se}
\def\handledareTre{Daniel Henriksson, danielh@cs.umu.se}

\def\inst{datavetenskap}
\def\dokumentTyp{Laborationsrapport}

\begin{document}
\begin{titlepage}
  \thispagestyle{empty}
  \begin{small}
    \begin{tabular}{@{}p{\textwidth}@{}}
      UMEÅ UNIVERSITET \hfill \today \\
      Institutionen för \inst \\
      \dokumentTyp \\
    \end{tabular}
  \end{small}
  \vspace{10mm}
  \begin{center}
    \LARGE{\preTitle} \\
    \huge{\textbf{\kurs}} \\
    \vspace{10mm}
    \LARGE{\title} \\
    \vspace{15mm}
    \begin{large}
        \namn, \mail \\
        \texttt{\pathtocode}
    \end{large}
    \vfill
    \large{\textbf{Handledare}}\\
    \mbox{\large{\handledareEtt}}
    \mbox{\large{\handledareTva}}
    \mbox{\large{\handledareTre}}
  \end{center}
\end{titlepage}

\pagestyle{fancy}
\rhead{\today}
\lhead{\namn, \mail}
\chead{}
\lfoot{}
\cfoot{}
\rfoot{}

\tableofcontents
\newpage

\rfoot{\thepage}
\pagenumbering{arabic}

\section{Problemspecifikation}
% Beskriv med egna ord vad uppgiften gick ut på. Är det någonting som
% varit oklart och ni gjort egna tolkningar så beskriv dessa.

Problemspecifikation finns i original på:\\
\verb!http://www.cs.umu.se/kurser/5DV085/HT08/labbar/lab2.html!

\section{Användarhandledning}
% Förklara var programmet och källkoden ligger samt hur man startar,
% kompilerar och använder det.
Programmet ligger i katalogen:\\
\texttt{\pathtocode}

Från denna katalog kompileras programmet med kommandot:

\verb!salt:~/edu/apjava/lab1> ant!

Källkoden ligger i underkatalogen \verb!src!.


\section{Systembeskrivning}
% Beskriv översiktligt hur programmet är uppbyggt och hur det löser
% problemet.

\section{Begränsningar}
% Vilka problem och begränsningar har din lösning av uppgiften? Hur
% skulle de kunna rättas till?

\section{Reflektioner}
% Var det något som var speciellt krångligt? Vilka problem uppstod och
% hur löste ni dem? Allmänna synpunkter. Hur skulle man kunna använda
% dessa metoder i andra mer omfattande system?

\section{Testkörningar}
% Noggranna testkörningar där man ser att programmet fungerar som det
% ska.

\section{Diskussion}
% Hur fungerade det att följa en kodkonvention? Vilka var fördelarna
% respektive nackdelarna?


\newpage
\appendix
\pagenumbering{arabic}
\section{Källkod}
% Källkoden ska finnas tillgänglig i er hemkatalog
% ~/edu/apjava/lab1/. Bifoga även utskriven källkod.
Härefter följer utskrifter från källkoden till denna laboration.

% \subsection{WriterOutputStream.java}\label{wos}
% \begin{footnotesize}
% \verbatiminput{../src/WriterOutputStream.java}
% \end{footnotesize}
\end{document}